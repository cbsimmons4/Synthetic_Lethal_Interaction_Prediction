\section{Introduction}
\label{sec:intro}
Genetic interactions are measurements of relationships between genes. For this project, we will be looking to predict synthetic lethal gene interactions (SL) and non-synthetic lethal gene interactions (non-SL). A synthetic lethal interaction occurs between two genes when the perturbation of either gene alone is viable but the perturbation of both genes simultaneously results in the loss of viability (Nigel, 2017). This means that two genes share a SL relationship if a combination of deficiencies in the expression of two or more genes leads to cell death, whereas a deficiency in only one of these genes does not. Otherwise, an interaction between two genes is considered to be Non-SL.

Predicting SL interactions is important because SLs are a mechanism we can use to target cancerous cells. Synthetic lethal genetic interactions with tumor-specific mutations may be exploited to develop anticancer therapeutics. One problem, however, is that predicting SLs can be hard to measure in humans. But, for simplicity, SLs can be measured in model organisms like yeast. In this paper, we look at two model species of yeast, Schizosaccharomyces pombe (Sp) and Saccharomyces cerevisiae (Sc). For both species we are start with a network of genetic interactions and a category for each interaction, SL or non-SL. The ultimate goal is to predict genetic interactions among species accurately. 

After predicting SL and non-SL among each different species, we also ask what would happen if we merged networks from different species. Then using the merged network, would we get better predictions of genetic interactions? The idea behind merging species networks is to use genes that are conserved across species. These genes are called homologs, and they serve similar functions in a pair of species. These genes can be seen as isomorphic to each other. As such, the merging of species is done by overlapping each separate network and connecting cross-species edges with homologous interactions. 

In order to help predict the category for interactions in each species network and the merged network, we used Node2vec to assist with machine learning. Given a network, Node2Vec is able to return a file with vector embedding for each node. 

Through machine learning on individual species networks, we aims to get a proof of concept for graph-vector embedding with genetic interaction networks. Can we make accurate predictions in the synthetic lethality of genes given only a species’ network? Then for the idea of merged species networks, we plan to determine if merging networks of two different species allows for better predictions of gene interactions. In essence, for cross-species networks, we looked at if merging the genetic makeup of two different species would allow us to better understand the gene interaction within the different species. For both the individual species and the cross-species networks, we ran the Node2Vec graph-vector embedding through various different machine learning models, seeing how each model deals with the embedment data, and we  determine which kind of models are best for making prediction on genetic interactions.   Our end goal was to find proof of concept for using graph-vector embedments to help assist with genetic interaction predictions and to make a baseline for merging networks of different species, hopefully to give some insight for future work in this domain.

A significant hurdle in our research was the initial lack of domain knowledge in biology and gene interactions. Throughout the process, we have been having to learning about genetic interactions. Max Leiserson from the University of Maryland - College Park Department of Computer Science, as well as Jason Fan, one of his PhD students, have been working on related areas. Therefore, throughout the semester, we have been collaborating with Jason, and he has been giving us insights on genetic interactions, how to best utilize various machine learning tools, how to merge networks, and ways to get better prediction for interactions. Also, through Jason’s help, we have gained insights to how machine learning on graph representation of genetic interactions can help suggest candidate treatments for cancer. The novel aspect of our approach to identifying SL gene interactions is our use of Node2Vec, a tool released in 2016 to learn patterns in graphs.

Machine learning on graphs, synthetic lethality related to cancer research, and cross-species merging are each not new topics on their own. As such, there are many related works that individually discuss each topic and build up to being able to effectively assist and act as prerequisites for our research. Collectively, the related topics were able to help pose our own questions for new ways of searching for candidate treatments for cancer. Therefore in section 2, we will discuss related works who were necessary prerequisites to completing our projects, as well as a detailed motivation for our own project. In section 3, we will review our methods for obtaining our results, and in section 4, we present our results. Ultimately, in sections 5, we suggest future work based off the groundwork of our research and the results we were able to achieve. 
